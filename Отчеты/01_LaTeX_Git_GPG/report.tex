\documentclass[a4paper, 12pt]{article}

\usepackage[utf8]{inputenc}
\usepackage[russian]{babel}
\usepackage{amsmath}
\usepackage{amsfonts}
\usepackage{amssymb}
\usepackage{graphicx}
\usepackage{hyperref}


\author{Баринов Дмитрий}
\title{Отчет по лабораторной работе}

\begin{document}


\begin{titlepage} \begin{center}

	\Large			
Санкт-Петербургский политехнический университет Петра Великого
			
	\vspace{0.2cm}	
Институт Информационных технологий и управления

	\vspace{2cm} \vfill \huge
Система верстки \TeX{} и рассширение \LaTeX{}

\vfill 

\begin{flushleft} \large \hangindent=8cm \hangafter=0
Выполнил: Баринов Д.С. гр. 53501/3 \hrulefill

Приняла: Вылегжанина К.Д. \hrulefill
\end{flushleft}

		
\vspace{2cm} \vfill \large
2015 г.
		

\end{center} \end{titlepage}

\newpage

\section{Система верстки \TeX{} и рассширение \LaTeX{}}

\subsection{Цель работы}

Изучение принципов верстки \TeX{}, создание первого отчета 

\subsection{Ход работы}

\subsubsection{Компиляция в командной строке}


Результатом работы Latex является файл в формате DVI - промежуточный формат для представления графического документа.

Команда: latex report.tex

Утилита xdvi позволяет отобразить DVI-файл.

Команда: xdvi report.dvi

Утилита pdflatex предназначена для трансляции tex файла напрямую в pdf.

Команда: pdflatex report.tex

\subsubsection{Оболочка TexMaker}

TexMaker - это удобная графическая оболочка для LaTex.

С его помощью можно быстро создать бланк документа, обратившись к помощнику и выбрав пункт меню "Быстрый старт". Далее пользователю будет предложено указать несколько основных настроек (тип документа, кодировка, используемые пакеты и т.п.) на основании которых будет сформирована начальная версия документа.

Также редактор предоставляет возможности полуавтоматического редактирования текста. Некоторые частоиспользуемые модификаторы вынесены в кнопки меню. Таким образом, изменить текст, например, сделав его жирным, можно также как и в более привычных редакторах. Для этого необходимо выделить текст и нажать на соответсвующую кнопку.

Главным преимуществом графического редактора является функция быстрой сборки. С ее помощью можно сразу же собрать документ и отобразить в соседней части экрана.

\subsubsection{Создание титульного листа, нескольких разделов, списка, несложной формулы}

Создание титульного листа и  разделов продемонстрировано выше.

Пример формулы и список приведены ниже:
\begin{enumerate}
\item Элемент 1
\item Формула 1: $$ x^2+y^2=z $$
\item Вложенный список
	\begin{itemize}
	\item Пункт 1
	\item Пункт 2
	\item Пункт 3
	\end{itemize}
\end{enumerate}

\subsubsection{Понятие классов документов, подключаемых модулей}

\LaTeX{} -файл должен начинаться с команды \textbackslash documentclass. 

Данная команда задает стиль оформления документа. В стандартый пакет входят такие стили как article, book, report, letter, proc. \\
Класс article удобно применять для статей, report - для более крупных статей, разбитых на главы, класс book - для книг.

Кроме того, имеется в \LaTeX имеется возможность включения стилевых пакетов, доспускающих задание своих личных стилевых опций. Включение стилвого пакета осуществляется командой \textbackslash usepackage.

Пример:

\begin{verbatim}
\usepackage[russian]{babel} % Пакет поддержки русского языка
\end{verbatim}

\subsection{Выводы}

\LaTeX{} наиболее популярный набор макрорасширений (или макропакет) системы компью-
терной вёрстки TEX, который облегчает набор сложных документов.
Пакет позволяет автоматизировать многие задачи набора текста и подготовки статей,
включая набор текста на нескольких языках, нумерацию разделов и формул, перекрёстные
ссылки, размещение иллюстраций и таблиц на странице, ведение библиографии и др.

\newpage
\section{Система контроля версий Git}

\subsection{Цель работы}

Изучить систему контроля версий Git, освоить основные приёмы работы с ней.

\subsection{Ход работы}

\begin{itemize}

\item{Получить содержимое репозитория
\begin{verbatim}git clone https://github.com/Kiker1/InfoSec2015.git 
\end{verbatim}}


\item{Добавить новую папку и первого файла под контроль версий
\begin{verbatim}
mkdir someDir
cd somedir
echo test>> somefile.txt
git add .
\end{verbatim}}

\item{Зафиксировать изменения в локальном репозитории
\begin{verbatim}
git commit -a -m "test" 
\end{verbatim}}

\item{Внести изменения в файл и просмотреть различия
\begin{verbatim}
echo test2 >> somefile.txt
git diff master:./somefile.txt ./somefile.txt
\end{verbatim}}

\item{Отменить локальные изменения
\begin{verbatim}
git reset HEAD ./somefile.txt
git checkout ./somefile.txt
\end{verbatim}}

\item{Внести изменения в файл и просмотреть различия
\begin{verbatim}
echo lalalal >> somefile.txt
git diff master:./somefile.txt ./somefile.txt
\end{verbatim}}

\item{Зафиксировать изменения в локальном репозитории, зафиксировать изменения в
центральном репозитории
\begin{verbatim}
git commit -a -m "update"
git push
\end{verbatim}}


\item{Поэкспериментировать с ветками
\begin{verbatim}
git branch test
git checkout test
git merge test
\end{verbatim}}
\end{itemize}

\subsection{Выводы}
Git -- это распределенная система контроля версий. Это означает, что кроме контроля версий git также предоставляет возможность удобной одновременной работы с репозиторием целой команды участников.

\newpage

\section{Cоздание электронных цифровых подписей c PGP}

\subsection{Цель работы}

Научиться создавать сертификаты, шифровать файлы и ставить ЭЦП.

\subsection{Ход работы}

\subsubsection{Изучить документацию, запустить графическую оболочку Kleopatra}

GPG4Win - набор программ, и руководств для осуществления подписи и шифровки документов.

Включает в себя:

\begin{enumerate}
\item GnuPG - программа шифрования
\item Kleopatra - графическая оболочка
\item GNU Privacy Assistant (GPA) - графическая оболочка (альтернатива)
\item GnuPG for Outlook (GpgOL) - расширение Microsoft Outlook для шифрования и подписи сообщений
\item GPG Explorer eXtension (GpgEX) - расширение Windows Explore для шифрования и подписи файлов используя контекстное меню
\item Claws Mail - полноценная программа для работы с почтой, предлагающая хорошую поддержку GnuPG
\end{enumerate}

\subsubsection{Создание ключевой пары}

Для создания ключевой пары необходимо выбрать пункт меню File - New Certificate - Create Personal OpenPGP key pair. 

\subsubsection{Экспорт сетрификата}

Для экспорта сертификата необходимо вызвать контекстное меню на сертификате и выбрать пункт Export Certificates. Данный сертификат был экспортирован в корень репозитория.

\subsubsection{Поставить ЭЦП на файл}

Используя графическую оболочку Kleopatra поставим подпись на файл messageToSuhinin.txt через меню File - Sign/Encrypt Files. Далее необходимо выбрать файл и указать тип действия (шифрование, подпись, оба). Полученный файл находится рядом с оригиналом  и называется messageToSuhinin.txt.sig.


\subsubsection{Получить чужой сертификат из репозитория}

Из каталога 
\url{https://github.com/vilegzhanina/InfoSecCourse2015/tree/master/%D0%9E%D1%82%D1%87%D0%B5%D1%82%D1%8B/01_LaTeX_Git_GPG}
были скачаны файлы karina.asc, myfirst.pdf, myfirst.pdf.sig. 

\subsubsection{Импортировать сертификат, подписать его}

Используя меню File - Import Certificates графической оболочки Kleopatra импортируем сертификат. Далее вызываем контекстное меню на сертификате и выбираем пункт Certify Cerificate. Этим действием мы подтверждаем свое доверие к источнику. В случае, если все прошло успешно, сертификат перейдет во вкладку Trusted Certificates.


\subsubsection{Проверить подпись}

Проверим подпись с помощью графической оболочки Kleopatra: выбрать пункт меню File - Decrypt/Verify File и указать файл подписи. 

В случае успешной проверки будет выведено сообщение The signature is valid and the certificate's validity is fully trusted. А также показаны данные подписавшего.  Для файла myfirst.pdf.sig данные выглядят следующим образом:

Signed on 2015-02-16 10:57 by k.vilegzhanina@gmail.com (Key ID: 0x391EA659).
The signature is valid and the certificate's validity is fully trusted.

\subsubsection{Взять сертификат кого-либо из коллег, зашифровать и подписать для него какой-либо текст, предоставить свой сертификат, убедиться, что ему удалось получить открытый текст, проверить подпись}

Для примера был взят сертификат Сухинина А.А. и импортирован согласно предыдущим пунктам. (файл cert.asc)

Далее, файл messageToSuhinin.txt был подписан и зашифрован для: 
\begin{itemize}
\item{barinovdmitrii@gmail.com} 
\item{suhininalex@gmail.com}
\item{k.vilegzhanina@gmail.com}
\end{itemize}

и отправлен Сухинину А.А., который подтвердил получение и успешную рашифровку и подпись.

\subsubsection{Предыдущий пункт наоборот}

Был взят файл, предоставленный Сухинином А.А. - messageToBarinov.txt.gpg.  Используя графический интерфейс Kleopatra данный файл был успешно расшифрован и проверена подпись. 

Текст сообщения: "Как поживает твоя сессия?" 

Signed on 2015-05-24 22:00 by suhininalex@gmail.com (Key ID: 0x2AB8346A).
The signature is valid and the certificate's validity is fully trusted.


\subsubsection{Использовании gpg через интерфейс командной строки}

Результат, полученный при помощи Kleopatra легко повторить используя терминал. 

\begin{itemize}


\item Генерация ключа происходит в диалоговом режиме после ввода команды

\begin{verbatim}
gpg --gen-key
\end{verbatim}

В процессе работы, мастер создания ключа запросит следующую информацию:

\begin{itemize}
\item{Тип ключа}
\item{Размер ключа}
\item{Время жизни ключа}
\item{Информацию о пользователе}
\item{Пароль для ключа}
\end{itemize}

\item Экспорт сертификата:

gpg --armor --export barinovdmitrii@gmail.com

\item Подпись файла:

gpg --detach-sign messageToSuhinin.txt

\item Проверка подписи:

gpg --verify messageToSuhinin.txt.sig


\end{itemize}

\subsection{Выводы}

В ходе работы были опробованы основные возможности GPG4win. Создание пары ключей, сохранение и экспорт сертификата, импорт сертификатов и их проверка, подпись и проверка подписи файлов, шифрование и расшифровка файлов. В ходе работы был опробован графический интерфейс Kleopatra и расширение GpgEX. Некоторые основные команды были также опробованы в консоли.

Общие впечатления от использования пакета Gpg - весьма положительные. Интерфейс Клеопатры удобен, а команды консоли интуитивно понятны и просты. 

\end{document}
